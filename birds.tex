\documentclass{article}
\usepackage{amsmath}
\usepackage{graphicx}
\usepackage{lettrine}
\usepackage{hyperref}
\setlength{\parindent}{0pt}
% \setlength{\parindent}{10pt}  
% Adjusts indent size

\begin{document}

\begin{titlepage}
    \centering
    \vspace*{1cm}
    
    \Huge
    \textbf{Planes and Birds: Minimizing energy}
    
    \vspace{0.5cm}
    
    \Large
    Tao Su
    
    \vspace{1.5cm}
    
    \large 
    \textit{HornorsCalc I Fall 2024 Project Final}
    
    \vfill

    \begin{figure}[h]
        \centering
        \includegraphics[width=1\textwidth]{coverPage.png}
    \end{figure}
\end{titlepage}

\newpage

\subsection*{\itshape \large The Introduction}


\lettrine[lines=2]{S}{mall} birds, such as finches, alternate between flapping their wings and gliding, a behavior essential to managing energy during flight. This project focuses on the relationship between a bird’s flight speed, the power required, and the average energy consumed, exploring how different speeds impact energy efficiency.

Using calculus, particularly what we learned in the chapter 4.7-- optimization by using the derivative and second derivative; we analyze the rate of change in power with respect to speed and the acceleration of energy consumption trends. The first derivative provides insight into how power requirements vary with speed, helping to identify critical points where power usage may peak or drop. The second derivative further refines this analysis by indicating whether these critical points represent energy-efficient speeds (minima) or less efficient speeds (maxima).

This approach draws on principles similar to those for fixed-wing aircraft, where energy and power requirements are essential to understanding flight dynamics. This project applies mathematical analysis to a biological context, using derivatives to derive meaningful insights about biologically relevant exploration of energy optimization in bird flight.

Through this study, we aim to deepen our understanding of the mathematical principles that govern flight and provide a meaningful analysis of energy efficiency in dynamic systems. By understanding derivatives and their applications, this project aligns with course objectives to apply calculus to real-world phenomena and develop strategies for energy optimization in natural systems.
\newpage


\subsection*{Part 1:}
\begin{figure}[h]
    \centering
    \includegraphics[width=0.5\textwidth]{bird.png}
    \caption{\small The trajectory of a flying bird.}
    \label{fig:bird}
\end{figure}

{\large \bfseries The power needed to propel an airplane forward at velocity \( v \) is: 
}\[
P = Av^3 + \frac{BL^2}{v}.
\]
{\large \bfseries where \( A \) and \( B \) are positive constants specific to the particular aircraft, and \( L \) is the lift—the upward force supporting the weight of the plane (or bird). Find the speed that minimizes the required power.}
\setlength{\parskip}{2em}

To determine the relationship between the power \(P\) and the speed \(v\), we can start with finding the derivative of \(P'\) with respect to \(v\).

The derivative of \( P \) with respect to \( v \) is:
\[
P'(v) = 3Av^2 - \frac{BL^2}{v^2}.
\]

To find the critical point of \( P(v) \), we set \( P'(v) = 0 \):
\[
P'(v) = 3Av^2 - \frac{BL^2}{v^2} = 0.
\]
Solving for \( v \), we get:
\[
v^4 = \frac{BL^2}{3A}.
\]
Therefore, we can find the speed \( V \) as:
\[
V = \sqrt[4]{\frac{BL^2}{3A}}.
\]

To determine whether this speed represents a minimum or maximum, we test the second derivative:
\[
P''(v) = 6Av + \frac{2BL^2}{v^3}.
\]

Since \( A \) and \( B \) are positive constants, and \( L^2 \) and \( v \) are always positive, we find that:
\[
P''(v) = 6Av + \frac{2BL^2}{v^3} > 0.
\]

Thus, when \( P'(v) = 0 \) and \( P''(v) > 0 \), \( P \) concaved up, indicating an absolute minimum. Therefore, when
\[
\text{the speed } V \text{ when power } p \text{ is minimized }  V_p = \sqrt[4]{\frac{BL^2}{3A}}
\]
the required power \( P \) is minimized.

\subsection*{Part 2:}
{\large \bfseries The speed found in Part 1 minimizes power, but a faster speed might use less fuel. The energy needed to propel the airplane/bird a unit distance is \( E = \frac{P}{v} \). At what speed is energy minimized?}

To find out the answer, we first need to know the relationship between \( E \) and \( v \). 

Fortunately, we know:
\[
P = Av^3 + \frac{BL^2}{v} \text{ from the beginning, and }
E = \frac{P}{v}
\]

So, we can easily derive the equation for \( E(v) \):
\[
E(v) = Av^2 + \frac{BL^2}{v^2}
\]
Next, let's take the derivative of \(E\) respect to \(v\):
\[
E'(v) = 2Av-\frac{2BL^2}{v^3}
\]
Same as part 1, we also need to find the critical point of \(E'(v)\):
\[\text{We set } E'(v) = 2Av-\frac{2BL^2}{v^3} = 0,\]
\[\text{ solve for }v, v^4=\frac{2BL^2}{2A}, v = \sqrt[4]{\frac{BL^2}{A}},\text{ and }V > 0.\]
Then we take the second derivative of \(E'\) respect to \(v\):
\[E''(v) = 2A + \frac{6BL^2}{v^4},
\text{ because } 2A > 0, \text{ and } \frac{6BL^2}{v^4} > 0, E''(v) > 0.\]
Therefore,
 \[\text{when the speed }V_e = \sqrt[4]{\frac{BL^2}{A}}, E\text{ has the absolute minimum,}\]
  the energy is minimized.

\subsection*{Part 3:}
{\large \bfseries How much faster is the speed for minimum energy than the speed for minimum power?
}\setlength{\parskip}{1em}

By observation, we found that both \(V_p\)(the speed \(V\) when power \(p\) is minimized) and \(V_e\) (the speed \(V\) when energy \(e\) is minimized) are 4th root, therefore, we can compare these two value by dividing \(V_e\) by \(V_p\):
\[\frac{V_e}{V_p}=\frac{\sqrt[4]{\frac{BL^2}{A}}}{\sqrt[4]{\frac{BL^2}{3A}}} = \sqrt[4]{3}\]

Therefore, we can conclude that, the speed \(V_e\) for minimum energy is \(\sqrt[4]{3}\) times faster than the speed for minimum power \(V_p\).

\subsection*{Part 4:}
\label{sec:part4}
{\large \bfseries In applying the equation of Part 1 to bird flight we split the term \(Av^3\)into two
parts: \(A_bv^3\) for the bird’s body and \(A_wv^3\) for its wings. Let \(x\) be the fraction of flying time spent in flapping mode. If \(m\) is the bird’s mass and all the lift occurs during flapping, then the lift is \(\frac{mg}{x}\) and so the power needed during flapping is:
\[P_\text{flap} = (A_b+A_w)v^3+\frac{B(\frac{mg}{x})^2}{v}\]
The power while wings are folded is \(P_\text{fold}=A_bv^3\). Show that the average power over an entire flight cycle is:
\[\overline{P}=xP_\text{flap}+(1-x)P_\text{fold} = A_bv^3+xA_wv^3+\frac{Bm^2g^2}{xv}\]}

To address this problem, first, let's break it down:

 The energy \(E\) for the whole cycle is actually generated by bird flapping its wings during the time of flapping stage, afterward, the bird stops flapping and folds its wings.

 But birds keeps flying, more acurately, it keeps gliding, during the time of folding stage, bird doesn't spend more energy, instead, it uses the remaining energy from the flapping stage.

Therefore, the total energy during the whole flying process is qual to the energy during flapping, and by the physics definition, the relationship between power \(P\) and energy \(E\) is:
\[\overline{P}=\frac{\Delta E}{\Delta t}\]
And the time during flapping is fraction \(x\), then the time during folding would be \(1-x\), therefore:
\[\text{the energy spends during flapping is } E_\text{flap} =P_\text{flap}\times  x \]
\[\text{the energy spends during folding is } E_\text{fold} = P_\text{fold}\times (1-x)\] 
\[\text{the total energy spent during the flying cycle would be: }E = E_\text{flap} +E_\text{fold} \]
And then,let's come back to the definition of power:
 \[\overline{P}=\frac{\Delta E}{\Delta t}=\frac{E}{x+(1-x)}=\frac{E_\text{flap}+E_\text{fold}}{1}=\frac{P_\text{flap}x + P_\text{fold}(1-x)}{1}\]
 Therefore:
\[\overline{P}=P_\text{flap}+P_\text{fold}(1-x)=((A_b+A_w)v^3+\frac{B(\frac{mg}{x})^2}{v})x+(A_bv^3)(1-x)\] 
\[= A_bv^3+xA_wv^3+\frac{Bm^2g^2}{xv}\]
Thus, we proved the hypothesis of average power during each flying cycle.


\subsection*{Part 5:}
{\large \bfseries For what value of \(x\) is the average power a minimum?
}\setlength{\parskip}{1em}

We've got the average power \(\overline{P}\) from the part 4, then let's take the derivative of \(\overline{P}\) respect to \(x\):
\[\overline{P}\text{ }'(x)= A_wv^3-\frac{Bm^2g^2}{x^2v}\]
and its second derivative:
\[\overline{P}\text{ }'' = \frac{2Bm^2g^2}{x^3v}\]
We found that, the second derivative of \(\overline{P}\) will be greater than \(0\), then we just need to find the \(x\) to make  \(\overline{P}\text{ }' = 0\):
\[\text{solve for }x \text{: }\overline{P}\text{ }'(x)= A_wv^3-\frac{Bm^2g^2}{x^2v} =0 \]
\[A_wv^3 = \frac{Bm^2g^2}{x^2v}, x = \sqrt{\frac{Bm^2g^2}{A_wv^4}} = \frac{mg}{v^2}\sqrt{\frac{B}{A_w}},\text{ and }x > 0\]
Therefore, when \(x = \frac{mg}{v^2}\sqrt{\frac{B}{A_w}}\), the average power is minimized.

\subsection*{Part 6:}
{\large \bfseries The average energy over a cycle is \(\overline{E} = \frac{\overline{P}}{v}\). What value of \(x\) minimizes \(\overline{E}\) ?}

Since we had \(\overline{P}\) from \hyperref[sec:part4]{part 4}, we can directly calculate \(\overline{E} = \frac{\overline{P}}{v}\):
\[\overline{E} = \frac{\overline{P}}{v}=  A_bv^2+xA_wv^2 + \frac{Bm^2g^2}{xv^2} \]
Then, let's take its derivative with respect to \(x\):
\[\overline{E}\text{ }' = A_wv^2-\frac{Bm^2g^2}{x^2v^2}\]

again, we need its second derivative with respect to \(x\) to determine its extremum:
\[\overline{E}\text{ }'' = \frac{2Bm^2g^2}{x^3v^2},\]
\[\text{both }Bm^2g^2 \text{ and } x^3v^2 \text{ are positive, therefore, } \overline{E} \text{ }'' > 0\]
So, when \(\overline{E}\text{ }' = 0\), \(\overline{E}\) has the absolute minimum value.

Next, we set \(\overline{E}\text{ }' = 0\), and solve for \(x\):
\[\overline{E}\text{ }' = A_wv^2-\frac{Bm^2g^2}{x^2v^2}, A_wv^2 = \frac{Bm^2g^2}{x^2v^2},\]
\[x^2=\frac{Bm^2g^2}{A_wv^4}, x = \frac{mg}{v^2}\sqrt{\frac{B}{A_w}},\text{ and }x > 0\]

Supprisingly, we got when \(x = \frac{mg}{v^2}\sqrt{\frac{B}{A_w}}\), it minimizes \(\overline{E}\), which is as the same as the value \(x\) minimizes the average power.

This is because \(\overline{P}= \frac{{\Delta \overline{E}}}{\Delta t}\), and \(\Delta t = x + (1-x) =1\), which make \(\overline{P} = \Delta \overline{E}\ = \overline{E}\).

\subsection*{Conclusion:}
{\large \bfseries Based on parts 5 and 6, what can you conclude if the bird flies slowly? What can you conclude if the bird flies faster and faster?}

We found the relationship between the flapping time \(x\) and the speed \(v\) to minimize the average power and energy from parts 5 and 6:
\[x = \frac{mg}{v^2}\sqrt{\frac{B}{A_w}}\]

When a bird has a limited amount of energy, it adjusts its speed based on its needs. To catch prey, the bird must fly faster, which increases its power output but shortens its flight cycle. After catching the prey, the bird slows down, which extends its flight cycle. If it doesn’t slow down, the bird will need to use more energy to keep flying.

This is similar to humans: when we sprint, we use a lot of energy to run faster. But when running a marathon, we slow down to conserve energy and make it to the finish line. A bird’s survival strategy is to fly quickly when necessary, like during a hunt, and slow down when it can, to save energy and stay efficient.

\textbf{\textit{But}}, the real world is far more complicated than our model, the bird's flight is influenced by many factors, such as the wind, the temperature, the humidity, and the bird's physical condition, etc. And if a bird flies too slow, it nees more energy to maintain its lift to stay aloft, if the bird flies too fast, the air resistance increases significantly, neither the extreme case is the optimal survive strategy for the bird, I really want to discuss about the optimal flying strategy, but the wisdom hides inside the body of a tiny bird is far beyond than my limit. Therefore, the bird's flight strategy is not only determined by the speed. These found make me agiain appreciate the complexity of the nature, and the beauty of the mathematics.

\end{document}